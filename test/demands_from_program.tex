

%=========================================
\subsection{Требования к функциональным характеристикам}
\subsubsection{Требования к составу выполняемых функций}
\begin{my_enumerate}
\item Интеграция в QEMU версии 2.10 или позднее.
\item Реализация алгоритма распределения регистров внутри блока трансляции.
\end{my_enumerate}

\subsubsection{Требования к организации входных и выходных данных}
Входными данными для работы алгоритма является массив инструкций для блока трансляции в формате внутреннего представления эмулятора QEMU. Для работы алгоритма необходима исполняемая программа, которая может быть запущена в эмуляторе QEMU. Входной файл исполняемой программы может быть создан в любой среде разработки на платформе которую поддерживает эмулятор QEMU, например х86\_64 с операционной системой Linux.

\begin{my_enumerate}
\item Файл программы должен представлять собой исполняемый файл предназначенный для запуска в userspace операционной системы Linux на архитектуре х86\_64.
\item Файл программы должен быть предоставлен в формате ELF.
\end{my_enumerate}

\medskip
Выходными данными для алгоритма являются коды команд для архитектуры х86\_64.


%=========================================
\subsection{Требования к временным характеристикам}
\begin{enumerate}
\item Работа алгоритма по распределению регистров должна завершаться в течении не более 0.1 секунды.
\end{enumerate}


%=========================================
\subsection{Требования к интерфейсу}
Требования к интерфейсу не предъявляются.


%=========================================
\subsection{Требования к надежности}
\subsubsection{Обеспечение устойчивого функционирования программы}
При некорректных входных параметрах должно отображаться сообщение об ошибке.
\subsubsection{Время восстановления после отказа}
Требования к восстановлению после отказа не предъявляются.
\subsubsection{Отказы из-за некорректных действий оператора}
Требования к отказу из-за некорректных действий оператора не предъявляются.
