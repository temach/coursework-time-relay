\documentclass[
%a4paper,12pt
encoding=utf8
]{./twoeskd}

% \usepackage{eskdappsheet}

% Packages required by doxygen
\usepackage[export]{adjustbox} % also loads graphicx
%\usepackage[utf8]{inputenc}
\usepackage{multicol}
\usepackage{multirow}
\usepackage{makeidx}
\usepackage{caption}
\usepackage{graphicx}

% NLS support packages
\usepackage[T2A]{fontenc}
\usepackage[russian]{babel}
\usepackage{pscyr}

% Font selection
\usepackage{courier}
\usepackage{amssymb}

% Page & text layout
% \usepackage{geometry}
% \geometry{%
%   a4paper,%
%   top=2.5cm,%
%   bottom=4.5cm,%
%   left=2.5cm,%
%   right=2.5cm%
% }
%\setlength{\emergencystretch}{15pt}
\setlength{\parindent}{0cm}
\setlength{\parskip}{0.2cm}

% Headers & footers
% \usepackage{fancyhdr}
% \pagestyle{fancyplain}
% \fancyhead[L]{\fancyplain{}{}}
% \fancyhead[C]{\fancyplain{}{\scriptsize\textbf{RU.17701729.509000 ТЗ 01-1-ЛУ}}}
% \fancyhead[R]{\fancyplain{}{}}
% \fancyfoot[L]{\fancyplain{}{}}
% \fancyfoot[C]{\fancyplain{}{}}
% \fancyfoot[R]{\fancyplain{}{}}

% debug to see the frame borders
% from https://en.wikibooks.org/wiki/LaTeX/Page_Layout
% \usepackage{showframe}

% Indices & bibliography
\usepackage{natbib}
\usepackage[titles]{tocloft}
\setcounter{tocdepth}{3}
\setcounter{secnumdepth}{5}

% change style of titles in \section{}
\usepackage{titlesec}
\titleformat{\section}[hang]{\huge\bfseries\center}{\thetitle.}{1em}{}
\titleformat{\subsection}[hang]{\Large\normalfont\raggedright}{\thetitle.}{1em}{\underline}
\titleformat{\subsubsection}[hang]{\large\normalfont\raggedright}{\thetitle.}{1pt}{}

% Packages for text layout in normal mode
% \usepackage[parfill]{parskip} % автоматом делает пустые линии между параграфами, там где они есть в тексте
% \usepackage{indentfirst} % indent even in first paragraph
\usepackage{setspace}	 % controls space between lines
\setstretch{1} % space between lines
\setlength\parindent{0.9cm} % size of indent for every paragraph
\usepackage{csquotes}% превратить " " в красивые двойные кавычки
\MakeOuterQuote{"}


% this makes items spacing single-spaced in enumerations.
\newenvironment{my_enumerate}{
\begin{enumerate}
  \setlength{\itemsep}{1pt}
  \setlength{\parskip}{0pt}
  \setlength{\parsep}{0pt}}{\end{enumerate}
}


% Custom commands
% configure eskd
\titleTop{
\textbf{
\Large ПРАВИТЕЛЬСТВО РОССИЙСКОЙ ФЕДЕРАЦИИ \\
НАЦИОНАЛЬНЫЙ ИССЛЕДОВАТЕЛЬСКИЙ УНИВЕРСИТЕТ \\
«ВЫСШАЯ ШКОЛА ЭКОНОМИКИ» } \\
\vspace*{0.2cm}
{\large Факультет компьютерных наук \\
Департамент программной инженерии \\
}
}
\titleDesignedBy{Студент группы БПИ 151 НИУ ВШЭ}{Абрамов А.M.}
\titleAgreedBy{%
\parbox[t]{7cm} {
Научный руководитель \\
доцент департамента \\
программной инженерии \\
канд. техн. наук. \\
}}{С. Л. Макаров}
\titleApprovedBy{
\parbox[t]{10cm} {
Академический руководитель \\
образовательной программы \\
«Программная инженерия» \\
профессор департамента программной \\
инженерии канд. техн. наук \\
}}{В. В. Шилов}
\titleName{ПРОГРАММА ДЛЯ УПРАВЛЕНИЯ И ВЗАИМОДЕЙСТВИЯ С РЕЛЕ ВРЕМЕНИ РВ-90}
\workTypeId{RU.02.07 51 01-1}
\titleSubname{Программа и методика испытаний}

% Custom packages
\usepackage{pdfpages}

\makeindex

%===== C O N T E N T S =====


\begin{document}
% Titlepage & ToC
\pagenumbering{roman}

% some water filling text, that is pointless but adds text
% \textbf{\Large Реферат} \\

Данная работа посвящена реализации программы для управления и взаимодействия с реле времени РВ-90. РВ-90 это устройство для автоматизации включения и выключения промышленных или домашних приборов в зависимости от времени.

Приведен сравнительный анализ других реле времени, которые используются для автоматизации в промышленности. Выявлены преимущества и недостатки характерные для подобных устройств.

Разработанная программа имеет клиент-серверную архитектуру и состоит из трех модулей: прошивка устройства, веб-интерфейс и Android приложение. Программа позволяет пользователю настроить РВ-90 для решения задач автоматизации. Прошивка устанавливается на само устройство и управляет периферией. Веб-интерфейс предоставляется пользователю через канал Wi-Fi и обеспечивает взаимодействие. Android приложение дублирует и расширяет возможности веб-интерфейса для владельцев Android.

В качестве основных технологий используемых для разработки прошивки выступают язык С, официальный SDK от компании Realtek для микроконтроллера RTL8711AM и GNU инструменты сборки программ для ARM процессоров. Для разработки веб-интерфейса использовались технологии CSS, HTML, Javascript, для сборки и оптимизации файлов использовался инструмент Gulp. Для разработки Android приложения использовались язык Java и среда разработки Android Studio.

Работа содержит \pageref{LastPage} страниц, 3 главы, 21 рисунок, 19 источников, 4 приложения.

\textbf{Ключевые слова:} автоматизация,  микроконтроллеры, Wi-Fi, реле времени.

\newpage

\textbf{\Large Annotation} \\

This work is focused on the implementation of the program for control and interaction with the time relay RV-90. RV-90 is a device for automation of industrial or home appliances as a function of time.

The comparative analysis of other time relays, which are used for automation in industry, is given. Advantages and disadvantages typical for such devices are revealed.

The developed program has a client-server architecture and consists of three modules: device firmware, web interface and an Android application. The program allows the user to configure the RV-90 to solve the task at hand. The firmware is meant to be installed on the device, it purpose is to control the peripherals. The web interface is provided to the user via a Wi-Fi channel and allows for user interaction. Android app duplicates and extends the capabilities of the web interface for the convenience of Android owners.

The cornersone technologies used for firmware development are the C programming language, official SDK from Realtek for the RTL8711AM microcontroller and the GNU ARM toolchain. The web interface is build using CSS, HTML, Javascript. Gulp was used to bundle and optimize the web interface files. The Android application was developed in Android Studio in the Java programming language. 

The work contains \pageref{LastPage} pages, 3 chapters, 21 drawings, 19 sources, 4 attachements.

\textbf{Keywords:} automation, microcontrollers, Wi-Fi, time relay.



\newpage
\pagenumbering{arabic}
\tableofcontents
% \pagenumbering{arabic}

% --- add my custom headers ---
\newpage
\section{Объект испытаний}
\subsection{Наименование}
Наименование: «Алгоритм для глобального распределения регистров в эмуляторе QEMU и его реализация». \\
Наименование на английском: «Algorithm for global allocation of registers in the QEMU emulator and its implementation». \\

\subsection{Область применения}
Реализованный алгоритм предназначен для включения в сборку программы QEMU на операционной системе Linux. Алгоритм может использоватся любым пользователем желающем ускорить работу эмулятора QEMU. Исходный код может использоваться в учебных целях как пример реализации алгоритма тесно взаимодействующего с внутренними механизмами QEMU.


\newpage
\section{Цель испытаний}
Цель проведения испытаний, - проверить, что разработанная программа соответствует требованиям к функциональности и надежности, изложенным в техническом задании к программе.


\newpage
\section{Требования к програмному изделию}


%=========================================
\subsection{Требования к функциональным характеристикам}
\subsubsection{Состав выполняемых функций}

\begin{my_enumerate}
\item Управление аппаратным разделением ресурсов. Организация семафоров и очередей на доступ к аппаратным ресурсам РВ-90 между несколькими задачами.
\item Управления состоянием коммутирующими реле (включено-выключено) в зависимости от времени, настроек системы и команд оператора.
\item Поддержка работы стека TCP/IP и управление сетью Wi-Fi. Поддержка нескольких клиентов сети.
\item Логирование и вывода отладочной информации.
\item Управление HTTP сервером. Обработка запросов поступающих от оператора. Передача файлов веб-приложения для запуска в веб-браузере на устройстве оператора. 
\item Анализ запросов и генерация ответов на запросы к API поступающие через HTTP сервер.
\item Сериализация и десериализация данных в/из формата JSON.
\item Взаимодействие с периферийными устройствами по протоколу I2C.
\item Чтение и программирование часов реального времени DS1307[4].
\item Синхронизация и поддержка корректного временем системы.
\item Управления файловой системой. Чтение и запись файлов, в частности организация хранения пользовательских данных и файлов веб-приложения.
\item Взаимодействие веб-интерфейса с сервером на микроконтроллере через AJAX.
\end{my_enumerate}

\subsubsection{Организация входных и выходных данных}
Алгоритм принимающий решение о коммутации электромагнитных реле в качестве входных данных получает либо команды пользователя, либо показания часов реального времени и составленную пользователем программу включений и выключений. Выходными данными для алгоритма являются управляющие сигналы на изменение состояния коммутирующих реле. 


\subsubsection{Требования к временным характеристикам}
Время реализации команды на включение или выключение коммутирующего реле при исполнении пользовательской программы не должно превышать 1 секунду.


%=========================================
\subsection{Требования к интерфейсу}
\begin{my_enumerate}
\item Отображения состояния реле в веб-интерфейсе.
\item Отображение параметров цикла из включений и выключений на один день.
\item Настройки конкретных дней для исполнения определенного цикла включений и выключений.
\item Отображения всех циклов и календаря циклов.
\item Предоставление инструкций для помощи новым пользователям.
\item Отображения ошибок и информационных сообщений с привязкой ко времени и степени важности сообщения.
\end{my_enumerate}


%=========================================
\subsection{Требования к надежности}
\subsubsection{Обеспечение устойчивого функционирования программы}
Программа должна устойчиво функционировать без аппаратных перезапусков в течении года.

\subsubsection{Время восстановления после отказа}
Требования к восстановлению после отказа не предъявляются.

\subsubsection{Отказы из-за некорректных действий оператора}
Программа не должна завершаться аварийно из-за некорректного взаимодействия оператора с веб-интерфейсом.


%=========================================
\subsection{Требования к условиям эксплуатации}
\subsubsection{Вид обслуживания}
Не требует каких-либо видов обслуживания.

\subsubsection{Численность и квалификация персонала}
Минимальное количество персонала, требуемого для работы: 1 оператор. Пользователь должен обладать практическими навыками работы с компьютером.

%=========================================
\subsection{Требования к составу и параметрам технических средств}
Для работы программы необходимо реле времени РВ-90. Для корректного функционирования веб-интерфейса требуется наличие у пользователя устройства с установленным браузером поддерживающим стандарты ECMAScript 5, HTML5, CSS3.


%=========================================
\subsection{Требования к информационной и программной совместимости}

Программа должна быть реализована на языке С и Rust. Управление задачами должно осуществляться с помощью FreeRTOS[2]. Для добавления интерактивности в веб-интерфейсе используется ECMAScript 5.


%=========================================
\subsection{Требования к упаковке}
Программа поставляется пользователю в виде заводской прошивки для реле времени РВ-90. 


\newpage
\section{Требования к програмной документации}
\subsection{Предварительный состав программной документации}
В обязательном порядке должны входить:
\begin{my_enumerate}
\item Техническое задание  (ГОСТ 19.201-78)
\item Пояснительная записка  (ГОСТ 19.404-79)
\item Руководство оператора  (ГОСТ 19.505-79)
\item Программа и методика испытаний (ГОСТ 19.301-79*)
\item Текст программы  (ГОСТ 19.401-78*)
\end{my_enumerate}



\newpage
\section{Средства и порядок испытаний}
%=========================================
\subsection{Параметры технических средств, используемых во время испытаний}
Для работы программы необходимо реле времени РВ-90. Компонент программы предназначенный для работы на платформе Android, должен работать лишь на планшетах и телефонах.
Для корректного функционирования веб-интерфейса требуется наличие у пользователя устройства с установленным браузером.


%=========================================
\subsection{Программные средства, необходимые для проведения испытаний}
Компонент программы предназначенный для работы на платформе Android, должен работать на планшетах и телефонах. 
Атрибут minSdkVersion объявляет минимальную версию, с которой совместимо приложение. Для реализации данного компонента была выбрана настройка minSdkVersion = 15. Данные настройки для минимальной версии SDK гарантируют что приложение будет работать на более чем 95\% процентах устройств подключающихся к Google Play Store.
Для корректного функционирования веб-интерфейса требуется наличие у пользователя устройства с установленным браузером поддерживающим стандарты ECMAScript-5, HTML-5, CSS-3. Приведенные браузеры являются поддерживаемыми платформами для работы одно-страничного веб-приложения:

\textbf{Мобильные браузеры}
\begin{my_enumerate}
\item Chrome for Android 73
\item Firefox for Android 66
\item UC Browser for Android 11.8
\item Android Browser 4.4
\item Android Browser 4
\item IE Mobile 11
\item iOS Safari 12
\item iOS Safari 8
\item Samsung Internet 9.2
\item Samsung Internet 7
\end{my_enumerate}

\textbf{Десктопные браузеры}
\begin{my_enumerate}
\item Chrome 74
\item Chrome 52
\item Firefox 66
\item Firefox 48
\item Edge 18
\item IE 10
\item Opera 58
\item Opera 39
\item Safari 12
\item Safari 9
\end{my_enumerate}


%=========================================
\subsection{Порядок проведения испытаний}
Испытания должны проводиться в следующем порядке:
\begin{my_enumerate}
\item Проверка требований к документации.
\item Проверка требований к интерфейсу.
\item Проверка требований к функциональным возможностям программы.
\item Проверка требований надежности.
\end{my_enumerate}


%=========================================
\subsection{Условия проведения испытаний}

\subsubsection{Требования к численности и квалификации персонала}
Минимальное количество персонала, требуемого для работы программы: 1 оператор.  Не требует от оператора высшего технического образования. 

\newpage
\section{Методы испытаний}
Испытания представляют собой процесс установления соответствия программы и
программной документации заданным требованиям.

\subsubsection{Проверка требований к документации}
Проверяеться наличие всех документов перечисленных в пyнкте 4.1 данного документа и их соответствие ГОСТ.

\subsection{Проверка требований к интерфейсу}
Требования к интерфейсу не предъявляются.

\subsection{Проверка требований к функциональным характеристикам}
Необходимо убедиться, что сборка проекта завершается успешно. Также необходимо убедиться, что алгоритм
действительно глобально распределяет регистры в блоке трансляции.


\subsection{Проверка требований к надежности}
Оператор должен проинспектировать сгенерированные коды команд для х86\_64 архитектуры и убедиться, что их выполнение соответствует корректному поведению программы, которая была запущена в эмуляторе.


\newpage
\section{Приложение 1. Терминология}
\subsection{Терминология}
\begin{description}


\item[Реле времени]

Прибор производственно-технического или бытового назначения в заданный момент времени выдающий определенный сигнал либо включающий/выключающий какое-либо оборудование через свое устройство коммутации электросети. 

\item[Часы реального времени]

Электронная схема, предназначенная для учёта хронометрических данных (текущее время, дата, день недели и др.).


\end{description}



%\newpage
%\section{Приложение 2. Схема интерфейса программы}
%\begin{figure}[h!]
    \centering
    \includegraphics[width=0.8\textwidth]{../screenshots/main_empty.png}
    \caption{Схема интерфейса}
\end{figure}


\newpage
\section{Приложение 2. Список используемой литературы}
\subsection{Список используемой литературы}
\begin{my_enumerate}

\item
Bellard Fabrice. QEMU, a Fast and Portable Dynamic Translator //
Proceedings of the Annual Conference on USENIX Annual Technical Conference. 2005.

\item
ГОСТ 19.103-77 Обозначения программ и программных документов. // Единая система программной документации. -М.: ИПК Издательство стандартов, 2001. \\

\item
ГОСТ 19.104-78 Основные надписи // Единая система программной документации. -М.: ИПК Издательство стандартов, 2001. \\

\item
ГОСТ 19.105-78 Общие требования к программным документам. // Единая система
программной документации. – М.: ИПК Издательство стандартов, 2001. \\


\end{my_enumerate}




% Index
\newpage
\eskdListOfChanges

% \phantomsection
% \addcontentsline{toc}{section}{Алфавитный указатель}
% \printindex

\end{document}
