%=========================================
\subsection{Параметры технических средств, используемых во время испытаний}
Для работы программы необходимо реле времени РВ-90. Компонент программы предназначенный для работы на платформе Android, должен работать лишь на планшетах и телефонах.
Для корректного функционирования веб-интерфейса требуется наличие у пользователя устройства с установленным браузером.


%=========================================
\subsection{Программные средства, необходимые для проведения испытаний}
Компонент программы предназначенный для работы на платформе Android, должен работать на планшетах и телефонах. 
Атрибут minSdkVersion объявляет минимальную версию, с которой совместимо приложение. Для реализации данного компонента была выбрана настройка minSdkVersion = 15. Данные настройки для минимальной версии SDK гарантируют что приложение будет работать на более чем 95\% процентах устройств подключающихся к Google Play Store.
Для корректного функционирования веб-интерфейса требуется наличие у пользователя устройства с установленным браузером поддерживающим стандарты ECMAScript-5, HTML-5, CSS-3. Приведенные браузеры являются поддерживаемыми платформами для работы одно-страничного веб-приложения:

\textbf{Мобильные браузеры}
\begin{my_enumerate}
\item Chrome for Android 73
\item Firefox for Android 66
\item UC Browser for Android 11.8
\item Android Browser 4.4
\item Android Browser 4
\item IE Mobile 11
\item iOS Safari 12
\item iOS Safari 8
\item Samsung Internet 9.2
\item Samsung Internet 7
\end{my_enumerate}

\textbf{Десктопные браузеры}
\begin{my_enumerate}
\item Chrome 74
\item Chrome 52
\item Firefox 66
\item Firefox 48
\item Edge 18
\item IE 10
\item Opera 58
\item Opera 39
\item Safari 12
\item Safari 9
\end{my_enumerate}


%=========================================
\subsection{Порядок проведения испытаний}
Испытания должны проводиться в следующем порядке:
\begin{my_enumerate}
\item Проверка требований к документации.
\item Проверка требований к интерфейсу.
\item Проверка требований к функциональным возможностям программы.
\item Проверка требований надежности.
\end{my_enumerate}


%=========================================
\subsection{Условия проведения испытаний}

\subsubsection{Требования к численности и квалификации персонала}
Минимальное количество персонала, требуемого для работы программы: 1 оператор.  Не требует от оператора высшего технического образования. 