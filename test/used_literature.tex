\subsection{Список используемой литературы}
\begin{my_enumerate}

\item
ГОСТ 19.102-77 Стадии разработки. //Единая система программной документации. -М.: ИПК Издательство стандартов, 2001. \\

\item
ГОСТ 19.201-78 Техническое задание. Требования к содержанию и оформлению // Единая система программной документации. -М.:ИПК Издательство стандартов, 2001. \\

\item
ГОСТ 19.101-77 Виды программ и программных документов
//Единая система программной документации. -М.: ИПК Издательство стандартов, 2.: 001. \\

\item
ГОСТ 19.103-77 Обозначения программ и программных документов. //Единая система программной документации. -М.: ИПК Издательство стандартов, 2001. \\

\item
ГОСТ 19.104-78 Основные надписи //Единая система программной документации. -М.: ИПК Издательство стандартов, 2001. \\

\item 
ГОСТ 19.105-78 Общие требования к программным документам. //Единая система
программной документации. – М.: ИПК Издательство стандартов, 2001. \\

\item
Omri Traub, Glenn Holloway, Michael D. Smith. Quality and Speed in Linear-scan Register Allocation. // In Proceedings of the ACM SIGPLAN 1998 Conference on Programming Language Design and Implementation, Montreal, QC, June 17-19, 1998: 142-151. \\

\item
Gregory Chaitin. Register allocation and spilling via graph coloring // In Proceedings of the 1982 SIGPLAN symposium on Compiler construction Pages 98-105  Boston, Massachusetts, USA — June 23 - 25, 1982, ISBN:0-89791-074-5 \\

\item 	Massimiliano Poletto, Vivek Sarkar. Linear Scan Register Allocation // In Journal
ACM Transactions on Programming Languages and Systems (TOPLAS) TOPLAS Homepage archive
Volume 21 Issue 5, Sept. 1999 Pages 895-913 \\

\item
Aho, Sethi, Ullman, Compilers: Principles, Techniques, and Tools // Addison-Wesley, 1986. ISBN 0-201-10088-6 \\


\end{my_enumerate}

