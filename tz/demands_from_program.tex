

%=========================================
\subsection{Требования к функциональным характеристикам}
\subsubsection{Состав выполняемых функций}

\begin{my_enumerate}
\item Управление аппаратным разделением ресурсов. Организация семафоров и очередей на доступ к аппаратным ресурсам РВ-90 между несколькими задачами.
\item Управления состоянием коммутирующими реле (включено-выключено) в зависимости от времени, настроек системы и команд оператора.
\item Поддержка работы стека TCP/IP и управление сетью Wi-Fi. Поддержка нескольких клиентов сети.
\item Логирование и вывода отладочной информации.
\item Управление HTTP сервером. Обработка запросов постипающих от оператора. Передача файлов веб-приложения для запуска в веб-браузере на устройстве оператора. 
\item Анализ запросов и генерация ответов на запросы к API поступающие через HTTP сервер.
\item Сериализация и десериализация данных в/из формата JSON.
\item Взаимодействие с периферийными устройствами по протоколу I2C.
\item Чтение и программирование часов реального времени DS1307[4].
\item Синхронизация и поддержка корректного временем системы.
\item Управления файловой системой. Чтение и запись файлов, в часности огранизация хранения пользовательских данных и файлов веб-приложения.
\item Взаимодействие веб-интерфейса с сервером на микроконтроллере через AJAX.
\end{my_enumerate}

\subsubsection{Организация входных и выходных данных}
Алгоритм принимающий решение о коммутации электромагнитных реле в качестве входных данных получает либо команды пользователя, либо показания часов реального времени и составленную пользовательем программу включений и выключений. Выходными данными для алгоритма являются управляющие сигналы на изменение состояния коммутирующих реле. 


\subsubsection{Требования к временным характеристикам}
Время реализации команды на включение или выключение коммутирующего реле при исполнении пользовательской программы не должно превышать 1 секунду.


%=========================================
\subsection{Требования к интерфейсу}
\begin{my_enumerate}
\item Отображения состояния реле в веб-интерфейсе.
\item Отображение параметров цикла из включений и выключений на один день.
\item Настройки конкретных дней для исполнения определенного цикла включений и выключений.
\item Отображения всех циклов и календаря циклов.
\item Предоставление инструкций для помощи новым пользователям.
\item Отображения ошибок и информационных сообщений с привязкой ко времени и степени важности сообщения.
\end{my_enumerate}


%=========================================
\subsection{Требования к надежности}
\subsubsection{Обеспечение устойчивого функционирования программы}
Программа должна устойчиво функционировать без аппаратных перезапусков в течении года.

\subsubsection{Время восстановления после отказа}
Требования к восстановлению после отказа не предъявляются.

\subsubsection{Отказы из-за некорректных действий оператора}
Программа не должна завершаться аварийно из-за некорректного взаимодействия оператора с веб-интерфейсом.


%=========================================
\subsection{Требования к условиям эксплуатации}
\subsubsection{Вид обслуживания}
Не требует каких-либо видов обслуживания.

\subsubsection{Численность и квалификация персонала}
Минимальное количество персонала, требуемого для работы: 1 оператор. Пользователь должен обладать практическими навыками работы с компьютером.

%=========================================
\subsection{Требования к составу и параметрам технических средств}
Для работы программы необходимо реле времени РВ-90. Для корректного функционирования веб-интерфейса требуется наличие у пользователя устройства с установленным браузером поддерживающим стандарты ECMAScript 5, HTML5, CSS3.


%=========================================
\subsection{Требования к информационной и программной совместимости}

Программа должна быть реализована на языке С и Rust. Управление задачами должно осуществляться с помощю FreeRTOS[2]. Для добавления интерактивности в веб-интерфейсе используется ECMAScript 5.


%=========================================
\subsection{Требования к упаковке}
Программа поставляется пользователю в виде заводской прошивки для реле времени РВ-90. 
