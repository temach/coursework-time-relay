\documentclass[
%a4paper,12pt
encoding=utf8
]{./twoeskd}

% \usepackage{eskdappsheet}

% apparently its better lo load PSCyr before babel and inputenc
\usepackage{pscyr}

% Packages required by doxygen
\usepackage[export]{adjustbox} % also loads graphicx
\usepackage{graphicx}
\usepackage[utf8]{inputenc}
\usepackage{multicol}
\usepackage{multirow}
\usepackage{makeidx}

% NLS support packages
\usepackage[T2A]{fontenc}
\usepackage[russian]{babel}

% Font selection
\usepackage{courier}
\usepackage{amssymb}

\setlength{\parindent}{0cm}
\setlength{\parskip}{0.2cm}

% debug to see the frame borders
% from https://en.wikibooks.org/wiki/LaTeX/Page_Layout
% \usepackage{showframe}

% Indices & bibliography
\usepackage{natbib}
\usepackage[titles]{tocloft}
\setcounter{tocdepth}{3}
\setcounter{secnumdepth}{5}

% change style of titles in \section{}
\usepackage{titlesec}
\titleformat{\section}[hang]{\huge\bfseries\center}{\thetitle.}{1em}{}
\titleformat{\subsection}[hang]{\Large\normalfont\raggedright}{\thetitle.}{1em}{\underline}
\titleformat{\subsubsection}[hang]{\large\normalfont\raggedright}{\thetitle.}{1pt}{}

% Packages for text layout in normal mode
% \usepackage[parfill]{parskip} % автоматом делает пустые линии между параграфами, там где они есть в тексте
% \usepackage{indentfirst} % indent even in first paragraph
\usepackage{setspace}	 % controls space between lines
\setstretch{1} % space between lines
\setlength\parindent{0.9cm} % size of indent for every paragraph
\usepackage{csquotes}% превратить " " в красивые двойные кавычки
\MakeOuterQuote{"}


% this makes items spacing single-spaced in enumerations.
\newenvironment{my_enumerate}{
\begin{enumerate}
  \setlength{\itemsep}{1pt}
  \setlength{\parskip}{0pt}
  \setlength{\parsep}{0pt}}{\end{enumerate}
}


% Custom commands
% configure eskd
\titleTop{
\textbf{
\Large ПРАВИТЕЛЬСТВО РОССИЙСКОЙ ФЕДЕРАЦИИ \\
НАЦИОНАЛЬНЫЙ ИССЛЕДОВАТЕЛЬСКИЙ УНИВЕРСИТЕТ \\
«ВЫСШАЯ ШКОЛА ЭКОНОМИКИ» } \\
\vspace*{0.2cm}
{\large Факультет компьютерных наук \\
Департамент программнoй инженерии \\
}
}
\titleDesignedBy{Студент группы БПИ 151 НИУ ВШЭ}{Абрамов А.M.}
\titleAgreedBy{%
\parbox[t]{7cm} {
Научный руководитель \\
доцент департамента \\
програмной инженерии \\
канд. техн. наук. \\
}}{С. Л. Макаров}
\titleApprovedBy{
\parbox[t]{10cm} {
Академический руководитель \\
образовательной программы \\
«Программная инженерия» \\
профессор департамента программной \\
инженерии канд. техн. наук \\
}}{В. В. Шилов}
\titleName{ПРОГРАММА ДЛЯ УПРАВЛЕНИЯ И ВЗАИМОДЕЙСТВИЯ С РЕЛЕ ВРЕМЕНИ РВ-90}
\workTypeId{RU.17701729.509000 T3 01-1}

\titleSubname{Техническое задание}


%===== C O N T E N T S =====


\begin{document}

% Titlepage & ToC
\pagenumbering{roman}

% some water filling text, that is pointless but adds text
% \textbf{\Large Реферат} \\

Данная работа посвящена реализации программы для управления и взаимодействия с реле времени РВ-90. РВ-90 это устройство для автоматизации включения и выключения промышленных или домашних приборов в зависимости от времени.

Приведен сравнительный анализ других реле времени, которые используются для автоматизации в промышленности. Выявлены преимущества и недостатки характерные для подобных устройств.

Разработанная программа имеет клиент-серверную архитектуру и состоит из трех модулей: прошивка устройства, веб-интерфейс и Android приложение. Программа позволяет пользователю настроить РВ-90 для решения задач автоматизации. Прошивка устанавливается на само устройство и управляет периферией. Веб-интерфейс предоставляется пользователю через канал Wi-Fi и обеспечивает взаимодействие. Android приложение дублирует и расширяет возможности веб-интерфейса для владельцев Android.

В качестве основных технологий используемых для разработки прошивки выступают язык С, официальный SDK от компании Realtek для микроконтроллера RTL8711AM и GNU инструменты сборки программ для ARM процессоров. Для разработки веб-интерфейса использовались технологии CSS, HTML, Javascript, для сборки и оптимизации файлов использовался инструмент Gulp. Для разработки Android приложения использовались язык Java и среда разработки Android Studio.

Работа содержит \pageref{LastPage} страниц, 3 главы, 21 рисунок, 19 источников, 4 приложения.

\textbf{Ключевые слова:} автоматизация,  микроконтроллеры, Wi-Fi, реле времени.

\newpage

\textbf{\Large Annotation} \\

This work is focused on the implementation of the program for control and interaction with the time relay RV-90. RV-90 is a device for automation of industrial or home appliances as a function of time.

The comparative analysis of other time relays, which are used for automation in industry, is given. Advantages and disadvantages typical for such devices are revealed.

The developed program has a client-server architecture and consists of three modules: device firmware, web interface and an Android application. The program allows the user to configure the RV-90 to solve the task at hand. The firmware is meant to be installed on the device, it purpose is to control the peripherals. The web interface is provided to the user via a Wi-Fi channel and allows for user interaction. Android app duplicates and extends the capabilities of the web interface for the convenience of Android owners.

The cornersone technologies used for firmware development are the C programming language, official SDK from Realtek for the RTL8711AM microcontroller and the GNU ARM toolchain. The web interface is build using CSS, HTML, Javascript. Gulp was used to bundle and optimize the web interface files. The Android application was developed in Android Studio in the Java programming language. 

The work contains \pageref{LastPage} pages, 3 chapters, 21 drawings, 19 sources, 4 attachements.

\textbf{Keywords:} automation, microcontrollers, Wi-Fi, time relay.



\newpage
\pagenumbering{arabic}
\tableofcontents

% --- add my custom headers ---
\newpage
\section{Введение}
\subsection{Введение}

В данном разделе представлено введение в предметную область, формулировка цели работы, а также новизна и практическая значимость работы.

Автоматизация - это технология, с помощью которой процесс или процедура выполняется с минимальным вовлечением человека. Некоторые процессы могут быть полностью автоматизированы. Автоматизация может быть востребована в самых разнообразных сценариях от термостата контролирующего бойлер загородного дома, до промышленной системы управления с набором исполнительных сигналов на выходе которые требуют высокой точности синхронизации по времени. Сложность решений варьируется начиная от простых элементов управления типа вкл/выкл и заканчивая алгоритмами, обрабатывающими сотни входящих сигналов каждую секунду. Преимущества автоматизации включают в себя экономию труда, экономию затрат на электроэнергию, экономию материальных затрат и повышение качества и точности исполнения процессов. В данной работе  рассматриваются простые системы автоматизации, которые характеризуются наличием часов реального времени (RTC) в качестве входа и от одного до четырех реле включения/выключения в качестве выходов. Такие устройства называются реле времени и обычно они делятся на 4 категории в зависимости от длительности временного интервала, для которого они позволяют устанавливать события включения/выключения: ежедневно, еженедельно, ежемесячно, ежегодно.

Устройства, которые в настоящее время доступны в нижнем и среднем ценовых  сегментах рынка, либо предоставляют очень ограниченный функционал, либо сложны в эксплуатации. Маленький LCD экран встречаемый на многих моделях серьезно ограничивает возможность вывода полезной информации пользователю. Для настройки, навигации по меню, а также для ввода всех числовых значений многие устройства предоставляют лишь кнопки на передней панели, что делает процесс медленным и подверженным ошибкам. Наконец, те редкие устройства, которые позволяют удаленное управление через компьютер или телефон, страдают от проблем с безопасностью соединения.

Тем не менее реле времени используются во многих отраслях, по оценкам компании ООО «Реле и Автоматика» годовой спрос на реле времени составляет примерно 20-30 тысяч[13].

Целью данной работы является разработка и реализация программы для управления и взаимодействия с реле времени РВ-90 (Реле и автоматика, Москва, Россия). РВ-90 это современное устройство для решения проблемы автоматизации с зависимостью от времени, которое направлено на то чтобы быть надежным и удобным в эксплуатации не компенсируя эти свойства дорогой ценой или сильно ограниченным функционалом. РВ-90 использует Wi-Fi и веб-технологии, которые позволяют пользователю со смартфоном подключиться и управлять реле с помощью отзывчивого, удобного интерфейса. На данный момент существует только инженерный дизайн устройства. Без программы для управления и взаимодействия данное реле использовать невозможно.



\newpage
\section{Основания для разработки}
\subsection{Документ, на основании которого ведется разработка}
Разработка программы ведется на основании приказа декана факультета компьютерных наук  
Национального исследовательского университета «Высшая школа экономики» 
\textnumero 2.3-02/1012-01 от 10.12.18
«Об  утверждении  тем,  руководителей  курсовых  работ  студентов
образовательной  программы  Программная  инженерия 
факультета 
компьютерных наук».


\subsection{Наименование темы разработки}
Наименование темы: «Программа для управления и взаимодействия с реле времени РВ-90». \\
Наименование темы на английском: «Program for Control and Monitoring of the Time Activated Relay RV-90». \\



\newpage
\section{Назначение разработки}
\subsection{Функциональное назначение}
Функциональным назначением разработки является предоставление пользователю возможности ускорить работу эмулятора QEMU.

\subsection{Эскплутационное назначение}
Реализованный алгоритм предназначен для включения в сборку программы QEMU на операционной системе Linux. Алгоритм может использоватся любым пользователем желающем ускорить работу эмулятора QEMU. Исходный код может использоваться в учебных целях как пример реализации алгоритма тесно взаимодействующего с внутренними механизмами QEMU.

\newpage
\section{Требования к программному изделию}


%=========================================
\subsection{Требования к функциональным характеристикам}
\subsubsection{Состав выполняемых функций}

\begin{my_enumerate}
\item Управление аппаратным разделением ресурсов. Организация семафоров и очередей на доступ к аппаратным ресурсам РВ-90 между несколькими задачами.
\item Управления состоянием коммутирующими реле (включено-выключено) в зависимости от времени, настроек системы и команд оператора.
\item Поддержка работы стека TCP/IP и управление сетью Wi-Fi. Поддержка нескольких клиентов сети.
\item Логирование и вывода отладочной информации.
\item Управление HTTP сервером. Обработка запросов поступающих от оператора. Передача файлов веб-приложения для запуска в веб-браузере на устройстве оператора. 
\item Анализ запросов и генерация ответов на запросы к API поступающие через HTTP сервер.
\item Сериализация и десериализация данных в/из формата JSON.
\item Взаимодействие с периферийными устройствами по протоколу I2C.
\item Чтение и программирование часов реального времени DS1307[4].
\item Синхронизация и поддержка корректного временем системы.
\item Управления файловой системой. Чтение и запись файлов, в частности организация хранения пользовательских данных и файлов веб-приложения.
\item Взаимодействие веб-интерфейса с сервером на микроконтроллере через AJAX.
\end{my_enumerate}

\subsubsection{Организация входных и выходных данных}
Алгоритм принимающий решение о коммутации электромагнитных реле в качестве входных данных получает либо команды пользователя, либо показания часов реального времени и составленную пользователем программу включений и выключений. Выходными данными для алгоритма являются управляющие сигналы на изменение состояния коммутирующих реле. 


\subsubsection{Требования к временным характеристикам}
Время реализации команды на включение или выключение коммутирующего реле при исполнении пользовательской программы не должно превышать 1 секунду.


%=========================================
\subsection{Требования к интерфейсу}
\begin{my_enumerate}
\item Отображения состояния реле в веб-интерфейсе.
\item Отображение параметров цикла из включений и выключений на один день.
\item Настройки конкретных дней для исполнения определенного цикла включений и выключений.
\item Отображения всех циклов и календаря циклов.
\item Предоставление инструкций для помощи новым пользователям.
\item Отображения ошибок и информационных сообщений с привязкой ко времени и степени важности сообщения.
\end{my_enumerate}


%=========================================
\subsection{Требования к надежности}
\subsubsection{Обеспечение устойчивого функционирования программы}
Программа должна устойчиво функционировать без аппаратных перезапусков в течении года.

\subsubsection{Время восстановления после отказа}
Требования к восстановлению после отказа не предъявляются.

\subsubsection{Отказы из-за некорректных действий оператора}
Программа не должна завершаться аварийно из-за некорректного взаимодействия оператора с веб-интерфейсом.


%=========================================
\subsection{Требования к условиям эксплуатации}
\subsubsection{Вид обслуживания}
Не требует каких-либо видов обслуживания.

\subsubsection{Численность и квалификация персонала}
Минимальное количество персонала, требуемого для работы: 1 оператор. Пользователь должен обладать практическими навыками работы с компьютером.

%=========================================
\subsection{Требования к составу и параметрам технических средств}
Для работы программы необходимо реле времени РВ-90. Для корректного функционирования веб-интерфейса требуется наличие у пользователя устройства с установленным браузером поддерживающим стандарты ECMAScript 5, HTML5, CSS3.


%=========================================
\subsection{Требования к информационной и программной совместимости}

Программа должна быть реализована на языке С и Rust. Управление задачами должно осуществляться с помощью FreeRTOS[2]. Для добавления интерактивности в веб-интерфейсе используется ECMAScript 5.


%=========================================
\subsection{Требования к упаковке}
Программа поставляется пользователю в виде заводской прошивки для реле времени РВ-90. 


\newpage
\section{Требования к программной документации}
\subsection{Предварительный состав программной документации}
В обязательном порядке должны входить:
\begin{my_enumerate}
\item Техническое задание  (ГОСТ 19.201-78)
\item Пояснительная записка  (ГОСТ 19.404-79)
\item Руководство оператора  (ГОСТ 19.505-79)
\item Программа и методика испытаний (ГОСТ 19.301-79*)
\item Текст программы  (ГОСТ 19.401-78*)
\end{my_enumerate}



\newpage
\section{Технико-экономические показатели}
\subsection{Ориентировочная экономическая эффективность}
Ориентировочная экономическая эффективность не рассчитывается.

\subsection{Экономические преимущества разработки}
Ориентировочные экономические преимущества разработки не рассчитывается.

\newpage
\section{Стадии и этапы разработки}

%=========================================
\subsection{Необходимые стадии разработки}

\subsubsection{Стадия разработки технического задания:}
\begin{my_enumerate}
\item Этап обоснования необходимости разработки программы:
    \begin{my_enumerate}
    \item постановка задачи.
    \item сбор исходных материалов.
    \end{my_enumerate}
\item Этап разработки и утверждения технического задания:
    \begin{my_enumerate}
    \item определение требований к алгоритму.
    \item определение стадий, этапов и сроков разработки программы и документации на нее.
    \item согласование и утверждение технического задания.
    \end{my_enumerate}
\end{my_enumerate}

\subsubsection{Стадия разработки технического проекта:}
\begin{my_enumerate}
\item Этап исследования уже существующих решений:
    \begin{my_enumerate}
    \item поиск уже созданных решений
    \item изучение их структуры и архитектуры
    \end{my_enumerate}
\item Этап разработки технического проекта:
    \begin{my_enumerate}
    \item разработка алгоритма
    \item разработка структуры и архитектуры частей алгоритма.
    \end{my_enumerate}
\item Этап утверждения технического проекта:
    \begin{my_enumerate}
    \item разработка плана мероприятий по разработке программы
    \item разработка пояснительной записки.
    \end{my_enumerate}
\end{my_enumerate}


\subsubsection{Стадия разработки рабочего проекта:}
\begin{my_enumerate}
\item Этап разработки программы:
    \begin{my_enumerate}
    \item непосредственное программирование и отладка алгоритма.
    \end{my_enumerate}
\item Этап разработки программной документации:
    \begin{my_enumerate}
    \item разработка следующих программных документов в соответствии с требованиями: техническое задание, пояснительная записка, руководство оператора, программа и методика испытания, текст программы, все в соответствии с требованиями ГОСТ 19.101-77.
    \end{my_enumerate}
\item Этап испытания программы:    
    \begin{my_enumerate}
    \item разработка, согласование и утверждение программы и методики испытаний.
    \item защита презентации, сдача разработанной документации.
    \item корректировка программы и программной документации по результатам защиты.
    \end{my_enumerate}
\end{my_enumerate}


%=========================================
\subsection{Сроки работ и исполнители}
Алгоритм должен быть разработан к 24 мая 2019 года, студентом группы БПИ151 Абрамовым Артемом.


\newpage
\section{Порядок контроля и приемки}
\subsection{Виды испытаний}
Контроль и приемка разработки осуществляются в соответствии с разработанным исполнителем и согласованным с заказчиком документом «Программа для управления и взаимодействия с реле времени РВ-90» Программа и методика испытаний по (ГОСТ 19.301-79*).

\subsection{Требования к приемке работы}
Акт приемки-сдачи программы между исполнителем и заказчиком в эксплуатацию происходит при выполнении указанных в настоящем документе функций и требований, при наличии документации к программе, выполненной в соответствии с требованиями настоящего технического задания.

\newpage
\section{Приложение 1. Терминология}
\subsection{Терминология}
\begin{description}


\item[Реле времени]

Прибор производственно-технического или бытового назначения в заданный момент времени выдающий определенный сигнал либо включающий/выключающий какое-либо оборудование через свое устройство коммутации электросети. 

\item[Часы реального времени]

Электронная схема, предназначенная для учёта хронометрических данных (текущее время, дата, день недели и др.).


\end{description}



%\newpage
%\section{Приложение 2. Схема интерфейса программы}
%\begin{figure}[h!]
    \centering
    \includegraphics[width=0.8\textwidth]{../screenshots/main_empty.png}
    \caption{Схема интерфейса}
\end{figure}


\newpage
\section{Приложение 2. Список используемой литературы}
\subsection{Список используемой литературы}
\begin{my_enumerate}

\item
Bellard Fabrice. QEMU, a Fast and Portable Dynamic Translator //
Proceedings of the Annual Conference on USENIX Annual Technical Conference. 2005.

\item
ГОСТ 19.103-77 Обозначения программ и программных документов. // Единая система программной документации. -М.: ИПК Издательство стандартов, 2001. \\

\item
ГОСТ 19.104-78 Основные надписи // Единая система программной документации. -М.: ИПК Издательство стандартов, 2001. \\

\item
ГОСТ 19.105-78 Общие требования к программным документам. // Единая система
программной документации. – М.: ИПК Издательство стандартов, 2001. \\


\end{my_enumerate}



% Index
\newpage
\eskdListOfChanges

% \phantomsection
% \addcontentsline{toc}{section}{Алфавитный указатель}
% \printindex

\end{document}
