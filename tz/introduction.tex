\subsection{Наименование программы}
Наименование программы: «Программа для управления и взаимодействия с реле времени РВ-90». \\
Наименование программы на английском: «Program for Control and Monitoring of the Time Activated Relay RV-90». \\


\subsection{Краткая характеристика}
Целью данной работы является разработка и реализация программы для управления и взаимодействия с реле времени РВ-90. РВ-90 это новое устройство предназначенное для автоматизации процессов с зависимостью от времени[1]. На данный момент РВ-90 только в виде схемотехника устройства, программное обеспечение отсутствует. В основе изделия используется микроконтроллер RTL8711AM[3] со встроенным функционалом Wi-Fi что позволяет пользователю со смартфоном подключиться и управлять реле с помощью удобного интерфейса. Програмное обеспечение должно сделать данное изделие надежным и удобным в эксплуатации при этом не урезая функционал устройства по сравнению с аналогами. Без программы для управления и взаимодействия данное реле использовать невозможно.