\subsection{РВ-90}
При первом включении РВ-90 создает открытую Wi-Fi сеть  с названием RV90-default.
Пользователь уже может взаимодействовать с системой через кнопки на корпусе. В частности кнопки для переключения состояния реле и для жестой перезагрузки системы.
 
Для подключения к сети Wi-Fi, которая создается РВ-90, пользователь использует стандартные средства предоставляемые его операционной системой.

\subsection{Подключение через веб-интерфейс}
После подключения с сети Wi-Fi, пользователь запрашивает в веб-браузере страницу по адресу http://10.0.0.1/. Это инициирует загрузку веб-приложения.

\subsection{Подключение через мобильное приложение}
После подключения к сети Wi-Fi, пользователь открывает мобильное приложение и выбирает пункт меню "Загрузить календари"

\subsection{Работа с временными отрезками}
Пользователь может создавать временные отрезки в течение которых реле будет находиться во включенном состоянии. Для создания нового отрезка пользователь нажимает кнопу "добавить временной отрезок" и задает время начала и окончания. Время указывается в 24 часовом формате. 


\subsection{Работа с циклами}
Пользователь может создавать дневные циклы. Для создания нового цикла пользователь нажимает кнопку "добавить дневной цикл". Цикл описывает распорядок включения и выключения реле на один день. Цикл представляет собой коллекцию из временных отрезков. Пользователь может изменить временные отрезки которые входят в цикл с помощью кнопки "Временные отрезки". Пользователь может изменить название и цвет, с которым цикл отображается в календаре, с помощью кнопки "Редактировать цикл".


\subsection{Работа с календарем}
Пользователь может добавить новый календарь нажав на кнопку "Добавить календарь".
Пользователь может редактировать календарь нажав на кнопку "Редактировать календарь"
В разделе редактирования пользователь может назначить каждому дню свой цикл.
Пользователь может назначить название для календаря.
Пользователь может перейти в обзорный режим, просмотреть весь календарь и посмотреть какие циклы назначены на какие дни. В режиме обзора каждый день подсвечивается цветом цикла который ему назначен. 

\subsection{Работа с состоянием реле}
Пользователь может заставить систему включить или выключить реле вне зависимости от запрограммированного календаря. Для этого он нажимает на кнопки ON/OFF в экране работы с реле.

\subsection{Работа с настройкой времени РВ-90}
Текущее время системы отображается оранжевым шрифтом в экране "Настройки времени".
Для синхронизации времени пользователь может нажать кнопку "Синхронизировать"

\subsection{Работа с настройкой Wi-Fi}
Текущее название и пароль Wi-Fi отображаются на экране "Нaстройка Wi-Fi".
Пользователь может ввести новые пароль и название в текстовые поля и передать их 
в РВ-90, с помощью кнопки "передать в реле". Для вступления настрок в силу требуется перезапустить реле.

\subsection{Просмотр событий}
Пользователь может посмотреть журнал событий которые проишодили в системе с момента его подключения в экране "Журнал событий".

\subsection{Окно помощи}
Пользователь может посмотреть сведения о программе и получить поддержку в пункте меню "Помощь".

\subsection{Завершение работы с программой}
Происходит при нажатии на кнопку "Закрыть".
