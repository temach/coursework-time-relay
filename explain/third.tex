\subsection{Реализация прошивки}
\subsubsection{SDK от Realtek}
\subsubsection{GNU GCC ARM Toolchain}
\subsubsection{Среда разработки}
\subsubsection{Инструменты для навигации SDK}
\subsubsection{Инструменты разработки и отладки от ARM}
\subsubsection{Инструменты создания образа файловой системы}
\subsubsection{Исправление ошибок в оффициальном SDK от Realtek}
\subsubsection{Цикл разработки}
Безопасность и надежность являются первостепенными задачами. На основе анализа соответствующей работы был разработан ряд руководящих принципов и решений. Во-первых, сеть Wi-Fi должна быть размещена самим РВ-90 (рис. 3). Это поможет снизить риск безопасности, связанный с распространением информации через Интернет. Во-вторых, прошивка будет написана на языке программирования Rust. Это ограничит количество ошибок связанных с переполнением буфера, стека и ряда других типичных для приложений написанных на языке C. Использование Rust также должно упростить тестирование, поскольку юнит- тесты являются частью языка, что будет способствовать поддержанию качество кода. Ожидается, что безопасность будет дополнительно повышена путем выбора руководящих принципов MISRA-C, которые могут применяться в контексте языка программирования Rust. 

Использование SoC на основе ARM означает, что Rust может использоваться в качестве языка программирования для выбора прошивки вместо C. необходимо также учитывать тот факт, что флеш-память ограничена 2 МБ. Часть этого пространства должна быть зарезервирована для прошивки, часть для веб-приложения и часть для пользовательских данных. Веб-приложение должно быть как можно меньше, чтобы поместиться во флеш-память вместе с изображениями, шрифтами, библиотеками и другими необходимыми данными. Программа для контроля и мониторинга РВ-90 должна создавать и поддерживать собственную сеть WiFi, чтобы устройство работало даже в зоне без подключения к интернету. Таймер RTC DS1307 имеет разрешение точности до миллисекунды, для того чтобы удовлетворить требованиям для точного времени.


\subsection{Реализация одно-страничного веб-приложения}
\subsubsection{Набор инструментов разработки}
\subsubsection{Используемые библиотеки}
\subsubsection{Система сборки}
\subsubsection{Цикл разработки}



\subsection{Реализация мобильного приложения}
\subsubsection{Инструменты разработки}
\subsubsection{Используемые библиотеки}
\subsubsection{Адаптированные проекты}