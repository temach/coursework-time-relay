Описание выбранных методов, моделей, алгоритмов решения задач.
Во-первых описывается платформа РВ-90, ее аппаратные возможности и компоненты.
Во-вторых с аппаратной точки зрения описываются устройства которые могут взаимодейстовать с системой.
В-третьих описывается архитектура программы, модулей (включая те, которые предназначены для передачи на устройство пользователя).
В-четвертых описывается модель взаимодействия между модулями системы. 

\subsection{ Аппаратная база }
Аппаратная база представлена в виде устройства РВ-90. Нет смысла рассматривать абсолютно все физические характеристики устройства. Далее будут рассмотренны те из них, которые влияют на реализацию программы. В основе устройства находится модуль RAK473. К GPIO вводам которого подключены: 3 выходящих штыря для программирования, питание 5 вольт, земля, шина I2C. К шине I2C подключены часы реального времени DS1307 и I2C расширитель портов PCA9555. К GPIO выводам расширителя подключены 4 светодиода, 3 push-down кнопки, 2 реле RT424005.

\subsection{ Разбор периферийных компонент для аппаратной базы }
Выбор компонентов влияет на архитектуру программы. В данной секции в больших деталях рассмотренны некоторые компоненты РВ-90 и их аналоги с целью понимания специфики каждого компонента. 

На корпусе имеются 3 кнопки, 2 из них для ручного переключения состояния каждого из двух реле, и одна для жесткой перезагрузки системы.
Кнопки не подключены ко входу прерывания расширителя. Программа должна периодически опрашивать текущее напряжение на кнопках, чтобы заметить нажатие. По умолчанию линии кнопок притянуты вверх, поэтому напряжение уровня земли соответствует нажатию. Программа должна применять фильтр к данным о состоянии кнопок, чтобы исключить ложные срабатывания, а также и двойные/тройные срабатывания из-за дребежзания контактов. 

Выбор I2C расширителя портов PCA9555 обоснован его низкой ценой и расширением на 16 GPIO портов, которых более чем хватает для контроля аналоговой периферии и возможностью частично задать адрес I2C  на этапе производства устройства для избежания конфликта адресов на шине. В качестве аналога существует более дешевый PCA9554 предоставляющий расширение на 8 GPIO портов и имеющий жестко зафиксированный адрес на шине. 

Выбор часов реального времени DS1374 имеет недостаток в виде высокой цены и много достоинств. 
Для взимодействия с микроконтроллером данные часы используют протокол I2C, вместе с другой периферией их можно подключить к общей шине I2C, что очень важно в условиях ограниченного колличества GPIO пинов  микроконтроллера.
Данные часы отсчитывают время в формате Unix-time. Формат Unix-time удобен для хранения, сравнения и передачи. Поскольку подразумевается что пользователь может создавать много событий включения/выключения требующих хранения времени, а память в системе ограничена, то важно иметь компактный формат для хранения времени. При взаимодействии с пользователем и при работе программы временные значения будут часто передаваться между различными програмными модулями, важно что формат Unix-time очень прост в представлении и следовательно удобен для передачи. При принятии решений система должна уметь возможность сравнивать и считать разницу между двумя временными значениями, время в формате Unix-time можно удобно и быстро сравнивать между собой (простое сравнение 32-битных целых чисел). Недостатком формата Unix-time в том что он не может отразить дополнительные високосные секунды [\url{https://stackoverflow.com/a/50910322/1073672}], а также в том что он может отображать только даты в промежутке с 1901 по 2038 год. Также в январе 2038 года данный формат будет подвержен ошибке из-за переполнения 32-битного целого числа. Программа для управления РВ-90 работает лишь с текущим годом, плюс-минус 10 лет, поэтому в данной работе ошибка переполенния никак не обрабатывается. Что касается дополнительных високосных секунд, то с 1970 по 2007 их было 23 [\url{http://pubs.opengroup.org/onlinepubs/9699919799/xrat/V4_xbd_chap04.html#tag_21_04_15}], поэтому за период эксплуатации устройства, состовляющий ориентировачно 2 года ожидается что их влияние будет несущественно. Тем не менее программа предоставляет пользователю возможность синхронизировать время в РВ-90 с другим устройством с целью устранения ошибки в отсчете времени на РВ-90. Часы DS1374 также имеют возможность генерировать программируемое прерывание.
Прерывания в определенный момент времени можно использовать для максимально точного контроля реле. Однако, в такой точности нет необходимости поскольку текущее время с точностью до секунды можно отсчитывать центральным процессором. Поэтому часы используются для борьбы с небольшой плавающей ошибкой, для этого программа синхронизируется с часами один раз в час. Также часы используются для бесперебойного отсчета времени когда устройство РВ-90 не подключено к электропитанию, для этого часы имеют отдельное питание от батарейки. Надо учитывать что часы используют 32-битное знаковое целое число для счетчика и подвержены ошибке переполнения счетчика в 2038 году. В данной работе эта ошибка не обрабатывается.  
В качестве аналога к часам DS1374 имеются часы DS1672, которые практически идентичны за исключением возможности генерации прерывания. 
Программа не имеет функционала для работы с часами использующими представление времени в формате BCD (Binary Coded Decimal). Часы в формате BCD удобны для вывода времени, кроме того они корректно отсчитывают текущую дату и время, принимая в расчет високосные года. В частности дополнительный день и секунды в високосный года, а также количество дней в каждом месяце в зависимости от текущего года. 
Пример часов работающих в формате BCD - DS1307, отличаются низкой ценой.


\subsubsection{ Разбор модуля РАК473 }
Производства компании Rak-Wireless [\url{http://docs.rakwireless.com/en/WIFI/RAK473/Hardware%20Specification/RAK473%C2%A0UART%C2%A0WiFi%C2%A0Module%C2%A0Specification%20V1.5.pdf}]. 
Включает в себя микроконтроллер RTL8711AM от Realtek, флеш-память GD25Q16C от GigaDevice подключаемую к микроконтроллеру через SPI интерфейс, Wi-Fi антенну, регуляторы и предохранители для GPIO пинов чипа. Схема GPIO выводов модуля представленя на рисунке.

\begin{figure}[h!]
    \centering
    \includegraphics[width=0.7\textwidth]{rak473_pinout.png}
    \caption{Рисунок 3. Схема GPIO выводов модуля RAK473.}
\end{figure}



Производства компании Rak-Wireless . 
Интегрируют RTL8711AM с флеш-памятью и антенной на одной печатной плате. У модуля есть документация.  Есть аналогичный модуль компании FN-link F11AMIM13\_B1.



Аппаратная база представлена в виде РВ-90. Это реле на основе микроконтроллера со встроенным Wi-Fi. Аппаратное обеспечение РВ-90 базируется на микроконтроллере RTL8711AM (Realtek, Hsinchu, Тайвань). Характеристиками данного аппаратного обеспечения являются его низкая стоимость, 2 МБ флеш-памяти, встроенная антенна Wi-Fi и чип DS1307 (Maxim Integrated, San Jose, California) для предоставления функций Real time clock (RTC). РВ-90 использует чип RTC в качестве входного сигнала и задает состояние двух выходящих реле как функцию от входного сигнала времени.


\subsubsection{ Ядро ARM Cortex-M3 }
Новое семейство процессоров, призванное занять новую для ARM нишу встраиваемых решений. В семействе присутствуют три значимых ядра.
Cortex-M0 с архитектурой ARMv6-M;
Cortex-M3 (опционально Memory Protection Unit) с архитектурой ARMv7-M;
Cortex-M4 (опционально Floating Point Unit) с архитектурой ARMv7E-M;

\subsubsection{ Микроконтроллер RTL8711AM }
Производства компании Realtek[7], это одно-кристальный микропроцессор нацеленный для изделий для Internet of Things. Он сочетает в себе ARM ядро на базе Cortex-M3, WLAN MAC и NFC в одном чипе. Он также предоставляет 19 GPIO. Конфигурация встроенной памяти RTL8711AM позволяет упростить и ускорить разработку приложений. Есть встроенная RAM и ROM. Выигрывает по характеристикам у более известного ESP8266. Периферия есть на борту. RAM и ROM имеют предопределенную карту памяти. Пользовательский код запускается из Flash память.
\subsubsection{ Модуль RAK473}
Производства компании Rak-Wireless. Интегрируют RTL8711AM с флеш-памятью и антенной на одной печатной плате. У модуля есть документация.  Есть аналогичный модуль компании FN-link F11AMIM13\_B1.
\subsubsection{Реле РВ-90}
Предоставляет светодиоды, исполнительные электромагнитные реле, платы с корпусом и RTC DS1307. Документация на каждую часть. I2C протокол. Типы реле. Выходы и их конфигурация.


\subsection{ Архитектура программы }
В результате проведенного исследования было принято решение о трехкомпонентной архитектуре. Первым компонентом является прошивка, устанавливаемая на РВ-90. Второй компонент представляет собой одно-страничное веб-приложение, которое будет храниться во флеш-памяти вместе с прошивкой.  Третий компонент - это Android-приложение, которое должно быть установлено на устройстве пользователя[12]. Каждый компонент создается с использованием своего стека технологий. Таким образом, программа состоит из трех отдельных компонентов, которые должны быть разработаны отдельно и должны будут взаимодействовать друг с другом. 

Конструкция аппаратной базы для РВ-90 является фиксированной и должна учитываться при проектировании архитектуры программы

\textbf{Три платформы}
\begin{my_itemize}
\item РВ-90
\item Браузер
\item Android
\end{my_itemize}


\begin{figure}[h!]
    \centering
    \includegraphics[width=0.7\textwidth]{three_platforms.png}
    \caption{Рисунок 3. Диаграмма взаимодействия между реле и клиентом.}
\end{figure}



\subsubsection{ Проектирование архитектуры прошивки для РВ-90 }
Что должна делать прошивка.
Безопасность и надежность являются первостепенными задачами. На основе анализа соответствующей работы был разработан ряд руководящих принципов и решений. Во-первых, сеть Wi-Fi должна быть размещена самим РВ-90 (рис. 3). Это поможет снизить риск безопасности, связанный с распространением информации через Интернет. Во-вторых, прошивка будет написана на языке программирования Rust. Это ограничит количество ошибок связанных с переполнением буфера, стека и ряда других типичных для приложений написанных на языке C. Использование Rust также должно упростить тестирование, поскольку юнит- тесты являются частью языка, что будет способствовать поддержанию качество кода. Ожидается, что безопасность будет дополнительно повышена путем выбора руководящих принципов MISRA-C, которые могут применяться в контексте языка программирования Rust. 

Использование SoC на основе ARM означает, что Rust может использоваться в качестве языка программирования для выбора прошивки вместо C. необходимо также учитывать тот факт, что флеш-память ограничена 2 МБ. Часть этого пространства должна быть зарезервирована для прошивки, часть для веб-приложения и часть для пользовательских данных. Веб-приложение должно быть как можно меньше, чтобы поместиться во флеш-память вместе с изображениями, шрифтами, библиотеками и другими необходимыми данными. Программа для контроля и мониторинга РВ-90 должна создавать и поддерживать собственную сеть WiFi, чтобы устройство работало даже в зоне без подключения к интернету. Таймер RTC DS1307 имеет разрешение точности до миллисекунды, для того чтобы удовлетворить требованиям для точного времени.

Прошивка устанавливается на РВ-90 в процессе производства.  Прошивка разбивается на несколько модулей. Модули разделены на основе функциональности, которая должна присутствовать в реле времени. Устройство каждого из модулей.

\begin{my_itemize}
\item Модуль управления многозадачностью и аппаратным разделением ресурсов.
\item Модуль для управления выходом реле (включено-выключено).
\item Модуль для работы стека TCP/IP и управления сетью Wi-Fi.
\item Модуль вывода отладочной информации.
\item Модуль управления HTTP сервером.
\item Модуль для анализа и генерации ответа на запросы API.
\item Модуль для анализа и генерации JSON.
\item Модуль для управления коммуникацией по протоколу I2C.
\item Модуль для связи с чипом DS1307 в режиме реального времени.
\item Модуль управления временем системы.
\item Модуль управления файловой системой.
\end{my_itemize}

\subsubsection{ Проектирование архитектуры веб-интерфейса}
Что должен делать каждый из модулей.

Веб-приложение будет написано в Typescript, чтобы свести к минимуму вероятность ошибок во время выполнения.

Пользовательский интерфейс реализован в веб-приложении. Оно предоставляет тот же функционал который можно встретить в других электронных реле времени. Оператор может: создать новый цикл включения/выключения и дать ему имя, назначить цикл определенному дню и установить его на повторение еженедельно или ежемесячно, установить исключения и переопределить циклы в определенные дни, получить обзор того, какие циклы выполняются в какие дни. Интерфейс интуитивно понятен и прост в использовании.

Одно-страничное веб-приложение хранится во флеш-памяти РВ-90 вместе с прошивкой. При подключении оно передается на телефон или планшет пользователя через браузер и в нем же и запускается. Веб-приложение можно разложить на следующие модули:

\begin{my_itemize}
\item Модуль для взаимодействия с сервером через AJAX.
\item Модуль для отображения состояния реле.
\item Модуль для настройки цикла включения / выключения на один день.
\item Модуль для настройки дней выполнения для цикла включения/выключения.
\item Модуль для отображения циклов включения/выключения и календарь циклов.
\item Модуль, чтобы помочь пользователю с началом работы.
\item Модуль для отображения ошибок и информационных сообщений.
\end{my_itemize}



\subsubsection{Проектирование архитектуры приложения Android}
На смартфоне установлено приложение для Android. Он используется для подключения к РВ-90, загрузки веб-приложения и запуска его. Он должен содержать следующие модули:

\begin{my_itemize}
\item Модуль для управления учетными данными Wi-Fi.
\item Модуль для подключения к сети Wi-Fi РВ-90.
\item Модуль для запроса, получения, запуска и остановки веб-приложения.
\end{my_itemize}


\subsection{Выводы по главе}
Рассмотрены архитектуры трех компонент программы. Спроектировано их взаимодействие, а также их взаимодействие с пользователем, спроектирована общая архитектура программы.

