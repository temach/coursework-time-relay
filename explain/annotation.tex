\textbf{\Large Реферат} \\

Данная работа посвящена реализации программы для управления и взаимодействия с реле времени РВ-90. РВ-90 это устройство для автоматизации включения и выключения промышленных или домашних приборов в зависимости от времени.

Приведен сравнительный анализ других реле времени, которые используются для автоматизации в промышленности. Выявлены преимущества и недостатки характерные для подобных устройств.

Разработанная программа имеет клиент-серверную архитектуру и состоит из трех модулей: прошивка устройства, веб-интерфейс и Android приложение. Программа позволяет пользователю настроить РВ-90 для решения задач автоматизации. Прошивка устанавливается на само устройство и управляет периферией. Веб-интерфейс предоставляется пользователю через канал Wi-Fi и обеспечивает взаимодействие. Android приложение дублирует и расширяет возможности веб-интерфейса для владельцев Android.

В качестве основных технологий используемых для разработки прошивки выступают язык С, официальный SDK от компании Realtek для микроконтроллера RTL8711AM и GNU инструменты сборки программ для ARM процессоров. Для разработки веб-интерфейса использовались технологии CSS, HTML, Javascript, для сборки и оптимизации файлов использовался инструмент Gulp. Для разработки Android приложения использовались язык Java и среда разработки Android Studio.

Работа содержит \pageref{LastPage} страниц, 3 главы, 21 рисунок, 19 источников, 4 приложения.

\textbf{Ключевые слова:} автоматизация,  микроконтроллеры, Wi-Fi, реле времени.

\newpage

\textbf{\Large Annotation} \\

This work is focused on the implementation of the program for control and interaction with the time relay RV-90. RV-90 is a device for automation of industrial or home appliances as a function of time.

The comparative analysis of other time relays, which are used for automation in industry, is given. Advantages and disadvantages typical for such devices are revealed.

The developed program has a client-server architecture and consists of three modules: device firmware, web interface and an Android application. The program allows the user to configure the RV-90 to solve the task at hand. The firmware is meant to be installed on the device, it purpose is to control the peripherals. The web interface is provided to the user via a Wi-Fi channel and allows for user interaction. Android app duplicates and extends the capabilities of the web interface for the convenience of Android owners.

The cornersone technologies used for firmware development are the C programming language, official SDK from Realtek for the RTL8711AM microcontroller and the GNU ARM toolchain. The web interface is build using CSS, HTML, Javascript. Gulp was used to bundle and optimize the web interface files. The Android application was developed in Android Studio in the Java programming language. 

The work contains \pageref{LastPage} pages, 3 chapters, 21 drawings, 19 sources, 4 attachements.

\textbf{Keywords:} automation, microcontrollers, Wi-Fi, time relay.

