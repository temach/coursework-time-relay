В работе была рассмотрена реализация программы для управления реле времени РВ-90. Была разработана трех-компонентная архитектура программы. Были подробно описаны платформы на которых будет запускаться каждый компонент. В частности для платформы РВ-90 был проведен подробный обзор аналогов для аппаратных компонент, с целью понимания специфики платформы и периферии. Основываясь на специфике платформы и на общей архитектуре программы, были разработаны архитектурные решения для каждого компонента. Функционал каждого компонента был разбит на несколько модулей и определены взаимосвязи между разными модулями системы. Также были определены взаимосвязи между тремя компонентами системы.
В  работе приведены необходимые теоретические и технические аспекты реализации компонент. 

В ходе выполнения ВКР были выполнены следующие задачи: 
\begin{my_enumerate}
\item Проанализированы различные современные методы решения задач автоматизации в зависимости от времени.
\item Предложена и реализована новая архитектура не требующая подключения к интернету и предоставляющая универсальный интерфейс в виде веб-приложения.
\item Исследованы возможности для обеспечения надежности функционирования компонента прошивки, в частности использование языка Rust. 
\item Разработаны три компонента программы. Прошивка для модуля RAK473, одно-страничное веб-приложение, а также мобильное приложение. 
\item Разработана  техническая  документация  программного  продукта,  приведенная  в приложениях А-Г.
\end{my_enumerate}

Дальнейшая работа включает в себя: 
\begin{my_enumerate}
\item Повышение надежности системы. Рассмотрение использование языка TypeScript для веб-приложения. Применение стандартов MISRA-C для разработки прошивки.
\item Написание юнит-тестов для всех компонентов системы.
\item Доработка веб-интерфейса для создания календаря, циклов и временных отрезков.
\item Расширение функционала Android приложения, например добавление шаблонов для создания календаря и циклов.
\item Создание мобильного приложения для iOS.
\end{my_enumerate}







