\subsection{Введение}

В данном разделе представлено введение в предметную область, формулировка цели работы, а также новизна и практическая значимость работы.

Автоматизация - это технология, с помощью которой процесс или процедура выполняется с минимальным вовлечением человека. Некоторые процессы могут быть полностью автоматизированы. Автоматизация может быть востребована в самых разнообразных сценариях от термостата контролирующего бойлер загородного дома, до промышленной системы управления с набором исполнительных сигналов на выходе которые требуют высокой точности синхронизации по времени. Сложность решений варьируется начиная от простых элементов управления типа вкл/выкл и заканчивая алгоритмами, обрабатывающими сотни входящих сигналов каждую секунду. Преимущества автоматизации включают в себя экономию труда, экономию затрат на электроэнергию, экономию материальных затрат и повышение качества и точности исполнения процессов. В данной работе  рассматриваются простые системы автоматизации, которые характеризуются наличием часов реального времени (RTC) в качестве входа и от одного до четырех реле включения/выключения в качестве выходов. Такие устройства называются реле времени и обычно они делятся на 4 категории в зависимости от длительности временного интервала, для которого они позволяют устанавливать события включения/выключения: ежедневно, еженедельно, ежемесячно, ежегодно.

Устройства, которые в настоящее время доступны в нижнем и среднем ценовых  сегментах рынка, либо предоставляют очень ограниченный функционал, либо сложны в эксплуатации. Маленький LCD экран встречаемый на многих моделях серьезно ограничивает возможность вывода полезной информации пользователю. Для настройки, навигации по меню, а также для ввода всех числовых значений многие устройства предоставляют лишь кнопки на передней панели, что делает процесс медленным и подверженным ошибкам. Наконец, те редкие устройства, которые позволяют удаленное управление через компьютер или телефон, страдают от проблем с безопасностью соединения.

Тем не менее реле времени используются во многих отраслях, по оценкам компании ООО «Реле и Автоматика» годовой спрос на реле времени составляет примерно 20-30 тысяч[1].

Целью данной работы является разработка и реализация программы для управления и взаимодействия с реле времени РВ-90 (Реле и автоматика, Москва, Россия). РВ-90 это современное устройство для решения проблемы автоматизации с зависимостью от времени, которое направлено на то чтобы быть надежным и удобным в эксплуатации не компенсируя эти свойства дорогой ценой или сильно ограниченным функционалом. РВ-90 использует Wi-Fi и веб-технологии, которые позволяют пользователю со смартфоном подключиться и управлять реле с помощью отзывчивого, удобного интерфейса. На данный момент существует только инженерный дизайн устройства. Без программы для управления и взаимодействия данное реле использовать невозможно.

